\documentclass{article}

\usepackage{graphicx}
\usepackage{amsmath}
\usepackage{amssymb}
\usepackage{caption}

\renewcommand{\figurename}{Figura}

\begin{document}

\begin{center}
    \Large Banjercito \\[24pt]
    \Large Diplomado en Ciencia de Datos
\end{center}

\vfill

\begin{flushleft}
    \LARGE \textbf{Notas del Curso} \\[12pt]
    \LARGE Módulo III | Semana 4 \\[24pt]
\end{flushleft}

\vfil

\begin{flushleft}
    Instructor: Alan Badillo Salas \\[12pt]
    Septiembre 2024 \\[24pt]
\end{flushleft}

\vfill

\section{Introducción}
En la semana 3 estudiamos el análisis de la varianza para un eje de datos segmentado por grupos. El principal objetivo fue crear una segmentación en el eje de datos que permitiera separar los datos en grupos. Si cada grupo se comportaba de manera similar a otro grupo, entonces la segmentación no reflejaría un cambio significativo entre los datos, y por lo mismo, la varianza de los grupos sería similar, implicando que los grupos no se contradicen respecto al eje.
\\[12pt]
De manera contraria, si un grupo o segmento tiene una media y varianza distinta a otro grupo, entonces, el grupo contradice el comportamiento del otro y esto significa que los grupos reflejan características diferentes, por lo que se aprueba una hipótesis de que los grupos de un mismo eje tienen diferente comportamiento bajo esa segmentación.
\\[12pt]
Este resultado se puede medir mediante la Prueba F de Fisher que mide la probabilidad de que dos segmentos de datos compartan una misma distribución en las colas, es decir, que compartan muestras que vuelvan ambiguo el determinar a qué segmento debería pertenecer el dato.
\\[12pt]
Esta semana estudiaremos la regresión simple y la regresión múltiple, las cuales nos permitirán encontrar la recta o el plano respectivamente que se ajusta mejor a nuestros datos, encontrando el valor de tendencia continuo sobre todo el eje.

\clearpage

% \tableofcontents

% \clearpage

\subsection{Contenido de la Semana 4}

Esta semana revisaremos los siguientes temas:

\begin{enumerate}
    \item Predicción de datos
    \begin{enumerate}
        \item Ejes independientes
        \item Eje dependiente
    \end{enumerate}
    \item Regresión lineal simple
    \begin{enumerate}
        \item Correlación entre dos ejes
        \item Línea de tendencia
        \item Regresión lineal $x \rightarrow y$
        \item Coeficientes $y = m \cdot x + b$
        \item Caso de estudio sobre la planta iris
        \item Caso de estudio sobre la supervivencia del titanic
    \end{enumerate}
    \item Regresión lineal múltiple
    \begin{enumerate}
        \item Correlación entre múltiples ejes
        \item Plano de tendencia
        \item Regresión lineal $x_1, x_2, \ldots, x_n \rightarrow y$
        \item Coeficientes $y = m_1 \cdot x_1 + m_2 \cdot x_2 + \ldots + m_n \cdot x_n + b$
        \item Caso de estudio sobre la planta iris
        \item Caso de estudio sobre la supervivencia del titanic
    \end{enumerate}
\end{enumerate}

\section{Análisis por Grupos}

Un eje de datos puede ser la edad, la estatura, el peso o el salario de una persona, pero también la altura de un tornillo en una línea de producción, la concentración de gas en un refresco o el monto de crédito aprobado en un préstamo. Un eje por sí mismo no aporta mucha información al análisis global de un estudio, para ello, podemos basarnos en otros ejes para segmentar los datos y encontrar cambios significativos entre los datos.
\\[12pt]
Por ejemplo, vamos a pensar que analizamos el salario de las personas, entonces encontramos que su salario promedio es de \$10,500.00 pesos, esto nos sirve para saber en promedio cuánto gana una persona. Pero ¿Qué pasa si segmentamos los datos en hombres y mujeres? Es decir, ¿Qué pasa si tomamos las muestras que corresponden a aquellas personas que son hombres y por otro lado separamos las muestras de personas que son mujeres?

\end{document}